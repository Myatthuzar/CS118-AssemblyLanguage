\def\VCDate{2016/03/28}\def\VCVersion{(Current)}
\documentclass{article}
\usepackage{fpic,color,pstcol,ProofPower,graphicx}
\begin{document}
Run doctex to produce the tex file, then pictures can be created using SML code.
Run docsml before running pdflatex...
\begin{GFT}{Text written to file run.sh}
\+doctex tryit.doc\\
\+docsml tryit.doc\\
\+poly < tryit.sml\\
\+pptexenv latex tryit.tex\\
\+dvips tryit.dvi\\
\+ps2pdf tryit.ps\\
\end{GFT}
\begin{GFT}{Bourne Shell}
\+chmod 755 run.sh\\
\end{GFT}
\fpic{pix}{
val b1 = box 2.0 4.0;
b1 hseq (b1 scaleXY (0.5,0.5));
}
\fpic{list}{
val cell = let val car = namePic dbox "car"
               val cdr = namePic dbox "cdr"
           in car vseq cdr end;

let val cells = (namePic cell "left") hseq (hspace 1.0) hseq
                (namePic cell "right")
    val source = cells pic "left" pic "cdr" pt "c"
    val target = cells pic "right" pic "car" pt "w"
in cells seq (bezier source (source ++ (1.0,0.0))
                     (target -- (1.0,0.0)) target
              withArrowStyle "->")
end;
}
\fpic{stackframe}{
val frame = 
 label "198" dbox vseq
            label "558" dbox vseq
            label "ret addr" dbox vseq
            label "old ebp" dbox;
frame;
}
(*
\fpic{more}{
infix 7 cellseq;
fun cell1 cellseq cell2 =
   let val cells = (group cell1) hseq (hspace 1.0) hseq (group cell2)
   in cells seq curvedharrow (cells nthpic 1 pic "cdr" pt "c")
                            (cells nthpic 3 pic "car" pt "w")
   end;
val cellseqlist = mkseqlist (op cellseq);
fun labelCar L = cell seq (L centeredAt (cell pic "car" pt "c"));
cellseqlist (map labelCar [text "A",text "B",text "C"]);
(*                            dcircle scaleTo (height (cell pic "car"), *)
(*                                              width (cell pic "car")), *)
(*                            cell scale 0.3]); *)
}
*)
\begin{GFT}{SML}
\+use "fpic.sml";\\
\+processTeXfile "tryit";\\
\+use "fpicpics.sml";\\
\+\\
\end{GFT}
\begin{GFT}{C source code written to file lab.c}
\+\#include <stdio.h>\\
\+int f(int x)\\
\+\{\\
\+   if(x > 0) return x + f(x-1);\\
\+   else return 0;\\
\+\}\\
\+int main()\\
\+\{\\
\+   printf("\%i\Backslash{}n",f(10));\\
\+\}\\
\end{GFT}


\end{document}
